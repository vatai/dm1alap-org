\documentclass[11pt,a4paper]{article}

\usepackage[magyar]{babel}
\usepackage[utf8]{inputenc}
\usepackage{t1enc}

\usepackage{amsmath}
\usepackage{amssymb}


\textwidth = 180mm
\hoffset = -30mm
\textheight = 265mm
\voffset = -30mm



\begin{document}

\thispagestyle{empty}

\begin{center}
\begin{large}
\noindent \textbf{Diszkrét Matematika 1.} Írásbeli vizsga, 2016. január 14. (90 perc)
\end{large}
\end{center}

{\noindent NÉV: \\ NEPTUN kód:\\ (Leendő) szakirány:\\}
\section{Alapvető fontosságú fogalmak}
A következő hat kérdésre 1-1 pont kapható. Ebből legalább 4 pontot kell szerezni.
\begin{enumerate}\setlength{\itemsep}{3cm}

\item Írja fel az alábbi három komplex számot trigonometrikus alakban: $i$, $-7$, $-2+2i$.

\item Mikor nevezünk egy $f:A\to B$ függvényt szürjektívnek? Szürjektív-e az $f(x) = -x+4$ függvény, ha $A=B=\mathbb{Z}$?

\item Húzza alá a kommutatívakat a következő (binér) műveletek közül (az alaphalmaz az egész számok halmaza): $(a,b)\mapsto a+b$; $(a,b)\mapsto a-b$; $(a,b)\mapsto ab$; $(a,b)\mapsto \max(a,b)$. (Itt $\max(a,b)$ az $a$ és $b$ számok maximumát jelöli.)\vspace{-1.2cm}

\item Hányféle sorrendben lehet leírni egy $k$ elemű halmaz elemeit?\vspace{-1cm}

\item Hogyan szól a logikai szita három halmaz uniójának elemszámára?

\item Soroljon fel 5 tulajdonságot a természetes számok körében az ,,osztója'' relációra.
\end{enumerate}

\newpage
\section{Definíciók, tételkimondások}
A következő nyolc kérdésre 1-1 pont kapható. 
\begin{enumerate}\setlength{\itemsep}{2.2cm}

\item Definiálja a komplex egységgyök fogalmát, és sorolja fel a negyedik egységgyököket (algebrai alakban).
\item Mikor nevezünk tranzitívnak egy relációt?
\item Definiálja részbenrendezésnél a maximális, illetve legnagyobb elem fogalmát. Következik-e valamelyik a másikból?
\item Adjon meg egy olyan $f\colon \mathbb{N}\to\mathbb{N}$ függvényt, mely nem szürjektív, de injektív.

\item Írja fel a polinomiális tételt.
\item Mondja ki az ismétléses variációk számára vonatkozó tételt.
\item Definiálja a maradékosztály és a redukált maradékosztály fogalmát.
\item Ismertesse az Euler-féle $\varphi$ függvény kiszámítására vonatkozó képletet (a prímfelbontásból, precízebben a kanonikus alakból kiindulva).



\end{enumerate}

\newpage
\section{Bizonyítások}
A következő három bizonyításra 3-3 pont kapható. Ebből legalább 3 pontot el kell érni (tételkimondásért nem jár pont).
Az összpontszám alapján a ponthatárok: 10-től 2-es, 14-től 3-as, 18-tól szóbelizhet a 4-es, illetve 5-ös osztályzatért.
\begin{enumerate}

\item Igazolja, hogy relációk kompozíciója asszociatív.
\item Mondja ki és igazolja az ismétléses permutációk számára vonatkozó tételt.
\item Mondja ki és igazolja az Euler--Fermat-tételt.

\end{enumerate}


\section{Szóbeli kiváltását lehetővé tevő opcionális tétel}
Ez a feladat maximálisan 5 pontot ér. Ha ebből legalább 3 pont megvan, és az összpontszám eléri a 20, illetve 24 pontot, akkor 4-es, illetve 5-ös érdemjegyet ajánlunk.
\begin{enumerate}

\item Kommutatív, illetve asszociatív-e az egész számok halmazán az $f:(a,b)\mapsto a + b -1$ művelet?
\item Asszociatív-e az egész számok halmazán a $g:(a,b)\mapsto ab + a + b$ művelet?
\item Asszociatív-e az egész számok halmazán az $(a,b)\mapsto ab + 2a + 2b$ művelet?
\item Milyen $k$ konstansokra lesz asszociatív az $(a,b)\mapsto ab + 2a + 2b + k$ művelet az egészek halmazán?
\item Igaz-e, hogy a $g$ művelet disztributív $f$-re?


\end{enumerate}


\end{document}

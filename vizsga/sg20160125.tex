\documentclass[11pt,a4paper]{article}

\usepackage[magyar]{babel}
\usepackage[utf8]{inputenc}
\usepackage{t1enc}

\usepackage{amsmath}
\usepackage{amssymb}


%\textwidth = 180mm
%\hoffset = -30mm
%\textheight = 265mm
%\voffset = -30mm



\begin{document}

\thispagestyle{empty}

\begin{center}
\begin{large}
\noindent \textbf{Sült galamb 1.} Írásbeli vizsga, 2016. január 25. (90 perc)
\end{large}
\end{center}

{\noindent NÉV: \\ NEPTUN kód:\\ (Leendő) szakirány:\\}
\section{Alapvető fontosságú fogalmak}
A következő hat kérdésre 1-1 pont kapható. Ebből legalább 4 pontot kell szerezni.
\begin{enumerate}

\item Definiálja a komplex abszolút érték fogalmát. Írjon fel 5 olyan
  komplex számot algebrai alakban, melyek abszolút értéke
  $5$. MO: $|z| = \sqrt{x^2+y^2}$ ahol
  $x=Re(z),y=Im(z)$, példák $|z|=5$-re:
  $\pm 5$, $\pm 5i$, $\pm 4 \pm 3i$ és $\pm 3 \pm 4i$
\item Definiálja binér reláció értelmezési tartományát (domain). Mi az
  alábbi reláció értelmezési tartománya:
  $R = \{ (x,y)\mid x,y\in \mathbb{N}, x> y\}$? MO:
  $dmn(R) =\{x : \exists y (x,y)\in R\}$, most $dmn(R)=\mathbb{N}^+$.
\item Húzza alá az asszociatívakat a következő (binér) műveletek közül
  (az alap\-halmaz az egész számok halmaza): $(a,b)\mapsto a+b$;
  $(a,b)\mapsto a-b$; $(a,b)\mapsto ab$; $(a,b)\mapsto
  \max(a,b)$. (Itt $\max(a,b)$ az $a$ és $b$ számok maximumát
  jelöli.) MO: A kivonás kivételével mind asszociatívak.
\item Hány 3 elemű rész\-halmaza van egy $k$ elemű halmaznak?
  $\binom{k}{3}$
\item Legalább hány számot kell kiválasztani a 10-nél kisebb
  természetes számok közül, hogy biztosan legyen köztük olyan, amely
  osztója a $18$-nak? MO: 6 ui. 18 osztói 1, 2, 3, 6, 9, stb és 0, 4,
  5, 7, 8 teljesítik a kritériumot, ezekből 5 van, és ha hat számot
  választunk akkor a skatulya elv alapján lesz legalább egy ami
  osztója 18-nak.
\item Definiálja a legnagyobb (kitüntetett) közös osztó fogalmát a
  természetes számok körében. MO: Az $a,b\in\mathbb{N}$ lnko-ja $d$ ha
  $d\mid a \land d \mid b$ és ha $d' \mid a \land d' \mid b$ akkor
  $d' \mid d$.
\end{enumerate}

\newpage
\section{Definíciók, tételkimondások}
A következő nyolc kérdésre 1-1 pont kapható. 
\begin{enumerate}

\item Definiálja a komplex egységgyök fogalmát, és sorolja fel a
  negyedik egységgyököket.  MO: $\varepsilon\in\mathbb{C}$ $n$-edik
  komplex egység\-gyök ha $\varepsilon^n=1$, ekkor
  $\exists k \in \{ 0,1,\ldots, n-1 \}$ hogy
  $\varepsilon=\cos\Bigl( \frac{2k\pi}{n}\Bigr) +i \sin\Bigl(
  \frac{2k\pi}{n}\Bigr)$. A negyedik komplex egység\-gyökök: $\pm 1$,
  $\pm i$.
\item Mikor nevezünk tranzitívnak egy relációt? MO: Egy
  $R\subset X \times X$ homogén binér relációt tranzitívnak nevezünk
  ha $\forall a,b,c$-re, $aRb \land bRc \Rightarrow aRc$.
\item Definiálja részbenrendezésnél a minimális, illetve legkisebb
  elem fogalmát. MO: Legyen $(X,\le)$ egy részben rendezett halmaz,
  $Y\subset X$, és $x\in Y$. $x$ minimális ha
  $\lnot \exists y ( y\in Y \land y < x)$
\item Adjon meg egy olyan $f\colon \mathbb{N}\to\mathbb{N}$ függvényt, mely nem injektív, de szürjektív.
\item Írja fel a (logikai) szita formulát.
\item Mondja ki az ismétléses variációk számára vonatkozó tételt.
\item Definiálja a maradékosztály és a redukált maradékosztály fogalmát.
\item Mondja ki Eukleidész tételét.



\end{enumerate}

\newpage
\section{Bizonyítások}
A következő három bizonyításra 3-3 pont kapható. Ebből legalább 3 pontot el kell érni (tételkimondásért nem jár pont).
Az összpontszám alapján a ponthatárok: 10-től 2-es, 14-től 3-as, 18-tól szóbelizhet a 4-es, illetve 5-ös osztályzatért.
\begin{enumerate}

\item Mondja ki és bizonyítsa a relációkompozíció asszociativitására vonatkozó tételt.
\item Mondja ki és igazolja az ismétléses kombinációk számára vonatkozó tételt.
\item Mondja ki és igazolja az egészek körében felbonthatatlanság és a prímtulajdonság egybeeséséről szóló tételt.

\end{enumerate}


\section{Szóbeli kiváltását lehetővé tevő opcionális tétel}
Ez a feladat maximálisan 5 pontot ér. Ha ebből legalább 3 pont megvan, és az összpontszám eléri a 20, illetve 24 pontot, akkor 4-es, illetve 5-ös érdemjegyet ajánlunk.

A binomiális tétel segítségével beláttuk, hogy $n\leq 1$ esetén $2^n = (1+1)^n =
\binom{n}{0} + \binom{n}{1}  + \binom{n}{2} + \cdots$, illetve azt, hogy
$0 = (1-1)^n = \binom{n}{0} - \binom{n}{1}  + \binom{n}{2} -
\cdots$. A két egyenletet összeadva látható, hogy a
Pascal-háromszög $n$-edik sorának minden második elemének összege,
vagyis $\binom{n}{0} + \binom{n}{2} + \binom{n}{4} + \cdots = 2^{n-1}$.

Az alábbiakban levezetünk egy képletet a Pascal háromszög minden
negyedik elemének összegére, vagyis a $\binom{n}{0} + \binom{n}{4} +
\binom{n}{8} + \cdots$ összegre.

\begin{enumerate}


\item Legyen $k$ egész szám. Igazoljuk, hogy ha $k$ osztható
4-gyel, akkor $i^{0} + i^{k} + i^{2k} + i^{3k} = 1$.


\item Legyen $k$ egész szám. Igazoljuk, hogy ha $k$ nem osztható
4-gyel, akkor $i^{0} + i^{k} + i^{2k} + i^{3k} = 0$.

\item Írjuk fel a binomiális tétel segítségével az $S_k = (1+i^k)^n$
összegeket a $k=0, 1, 2, 3$ értékekre (az $i^k$ alak helyett a konkrét
értékkel dolgozzunk).

\item Felhasználva az első két pontot, az $S_0+S_1 + S_2 + S_3$ összeg
vizsgálatával adjunk képletet az $\binom{n}{0} + \binom{n}{4} +
\binom{n}{8} + \cdots$ összegre.

\item Milyen $n$ értékekre lesz igaz, hogy $\binom{n}{0} + \binom{n}{4} +
\binom{n}{8} + \cdots = 2^{n-2}$?



\end{enumerate}


\end{document}

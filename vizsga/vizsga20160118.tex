\documentclass[11pt,a4paper]{article}

\usepackage[magyar]{babel}
\usepackage[utf8]{inputenc}
\usepackage{t1enc}

\usepackage{amsmath}
\usepackage{amssymb}


\textwidth = 180mm
\hoffset = -30mm
\textheight = 265mm
\voffset = -30mm



\begin{document}

\thispagestyle{empty}

\begin{center}
\begin{large}
\noindent \textbf{Diszkrét Matematika 1.} Írásbeli vizsga, 2016. január 18. (90 perc)
\end{large}
\end{center}

{\noindent NÉV: \\ NEPTUN kód:\\ (Leendő) szakirány:\\}
\section{Alapvető fontosságú fogalmak}
A következő hat kérdésre 1-1 pont kapható. Ebből legalább 4 pontot kell szerezni.
\begin{enumerate}\setlength{\itemsep}{3cm}

\item Írja fel a szorzásra vonatkozó Moivre-azonosságot. \vspace{-.5cm}
\item Mik a kvantorok, és mi a jelentésük?
\item Adjunk meg (rendezett párok halmazaként) három tranzitív relációt az $\{1, 2, 3\}$ alaphalmazon.
\item Hányféle dobás lehetséges 4 teljesen egyforma dobókockával?
\item Húzza alá a megoldható kongruenciákat: $21x\equiv 7\pmod 14$;  $30x\equiv 4\pmod 5$;  $11x\equiv 4\pmod {10}$;  $4132x\equiv 1\pmod {4133}.$\vspace{-1.5cm}
\item Mikor mondjuk az egészek körében, hogy két szám kongruens egymással modulo $m$?
\end{enumerate}

\newpage
\section{Definíciók, tételkimondások}
A következő nyolc kérdésre 1-1 pont kapható. 
\begin{enumerate}\setlength{\itemsep}{2.2cm}

\item Írja fel a negyedik egységgyököket \emph{trigonometrikus} alakban.
\item Definiálja a részbenrendezést.
\item Adjon meg egy olyan $f\colon \mathbb{N}\to \mathbb{N}$ függvényt, mely szürjektív, de nem injektív.
\item Definiálja a disztributivitást, és mutasson példát.

\item Bontsa fel a zárójelet a binomiális tétel segítségével: $(2x-3)^5$.
\item Hány olyan 7 hosszú sorozat képezhető a latin ábécé 26 eleméből, melyben csupa különböző betű szerepel (nem kell kiszámolni a konkrét számot, elég csak a képlet)?
\item Hogyan definiáljuk az Euler-féle $\varphi$ függvényt?
\item Mi a $47^{1281}$ utolsó két számjegye tízes számrenszerben?


\end{enumerate}

\newpage
\section{Bizonyítások}
A következő három bizonyításra 3-3 pont kapható. Ebből legalább 3 pontot el kell érni (tételkimondásért nem jár pont).
Az összpontszám alapján a ponthatárok: 10-től 2-es, 14-től 3-as, 18-tól szóbelizhet a 4-es, illetve 5-ös osztályzatért.
\begin{enumerate}

\item Mondjon ki és bizonyítson be a halmazok komplementerének tulajdonságai közül 5-öt.
\item Mondja ki és igazolja az ismétlés nélküli kombinációk számáról szóló állítást.
\item Mondjon ki és igazoljon az oszthatóság tulajdonságai közül 8-at a természetes számok körében.

\end{enumerate}


\section{Szóbeli kiváltását lehetővé tevő opcionális tétel}
Ez a feladat maximálisan 5 pontot ér. Ha ebből legalább 3 pont megvan, és az összpontszám eléri a 20, illetve 24 pontot, akkor 4-es, illetve 5-ös érdemjegyet ajánlunk.

Legyen $X$ az $\mathbb{N}$ véges részhalmazainak halmaza. Ez
részbenrendezett halmaz a ,,részhalmaza'' relációval (legkisebb eleme
az üres halmaz). Legyen $Y=\mathbb{N}$, ezen az ,,osztója'' relációt
tekintve szintén egy részbenrendezést kapunk ($a\preccurlyeq b$, ha $a$
osztója $b$-nek).

\begin{enumerate}

\item Mi $Y$ legkisebb eleme?

\item Legyen $p_0 = 2, p_1 = 3, p_2 = 5, \ldots$ a prímszámok
sorozata. Tekintsük a következő $f \colon X \to Y$ leképezést:
$f(A) = \prod_{k\in A} p_k$. Mennyi lesz $f(\{0, 1, 2\})$, illetve
$f(\{1, 3, 5\})$?

\item Injektív, szürjektív, illetve bijektív függvény-e $f$? Indokoljunk.

\item Igaz-e, hogy $f$ monoton növő, illetve szigorúan monoton növő?

\item Legyen $A$ egy $n$ elemű halmaz. Hány eleme van az $\{y\in Y\mid
y\preccurlyeq f(A)\}$ halmaznak?

\end{enumerate}


\end{document}

\documentclass[11pt,a4paper]{article}

\usepackage[magyar]{babel}
\usepackage[utf8]{inputenc}
\usepackage{t1enc}

\usepackage{amsmath}
\usepackage{amssymb}


\textwidth = 180mm
\hoffset = -30mm
\textheight = 265mm
\voffset = -30mm



\begin{document}

\thispagestyle{empty}

\begin{center}
\begin{large}
\noindent \textbf{Diszkrét Matematika 1.} Írásbeli vizsga, 2015. decmber 17. (90 perc)
\end{large}
\end{center}

{\noindent NÉV: \\ NEPTUN kód:\\ (Leendő) szakirány:\\}
\section{Alapvető fontosságú fogalmak}
A következő hat kérdésre 1-1 pont kapható. Ebből legalább 4 pontot kell szerezni.
\begin{enumerate}\setlength{\itemsep}{3cm}

\item Adja meg egy általános $a+bi$ algebrai alakú (nem nulla) komplex szám reciprokának képletét algebrai alakban, illetve konkrétan a $2-3i$ szám reciprokát. \vspace{-.5cm}
\item Írja fel az implikáció igazságtáblázatát.
\item Adja meg az összes olyan  $X\subseteq \{1, 2 \}$ halmazt, melyre az  $R = \{(a, b)\mid a,b\in X, a\leq b \}$ reláció tranzitív.
\item Hány különböző 5 hosszú sorozat képezhető 3 darab A, 2 darab B és 2 darab C betű felhasználásával?
\item Húzza alá a megoldható kongruenciákat: $3x\equiv 2\pmod 5$;  $22x\equiv 4\pmod 33$;  $14x\equiv 4\pmod {10}$;  $11x\equiv 1\pmod {111}$\vspace{-1.5cm}
\item Definiálja a felbonthatatlan szám fogalmát a természetes számok körében.
\end{enumerate}

\newpage
\section{Definíciók, tételkimondások}
A következő nyolc kérdésre 1-1 pont kapható. 
\begin{enumerate}\setlength{\itemsep}{4.2cm}

\item Mondja ki a halmazunió 3 tulajdonságát.
\item Mikor nevezünk trichotomnak egy relációt?
\item Definiálja a felső korlát fogalmát.
\item Mikor nevezünk szigorúan monoton növőnek egy $f:X\to Y$ függvényt? Milyen $X$, $Y$ halmazok esetén beszélhetünk erről?

\newpage
\item Hányféleképpen oszthatunk ki 10 darab 100 Ft-ost 3 ember között, ha az emberek különbözőek, és csak az számít, ki mennyi pénzt kapott (az összes pénzt kiosztjuk, és lehet olyan is, hogy valaki egyáltalán nem kap).
\item Hogyan tudjuk a logikai szita segítségével három véges halmaz uniójának elemszámát megbecsülni?
\item Mikor mondjuk az egészek körében, hogy $a$ és $b$ kongruensek modulo $m$? Hogy jelöljük? 
\item Mit nevezünk redukált maradékrendszernek modulo $m$? Adjon példát az $m=10$ esetben.



\end{enumerate}

\newpage
\section{Bizonyítások}
A következő három bizonyításra 3-3 pont kapható. Ebből legalább 3 pontot el kell érni (tételkimondásért nem jár pont).
Az összpontszám alapján a ponthatárok: 10-től 2-es, 14-től 3-as, 18-tól szóbelizhet a 4-es, illetve 5-ös osztályzatért.
\begin{enumerate}

\item Mondja ki és bizonyítsa be a komplex $n$-edik gyökvonásról szóló állítást (Moivre-képlet).
\item Mondja ki és igazolja a binomiális tételt.
\item Mondja ki az Euler--Fermat-tételt, majd igazolja is.

\end{enumerate}


\section{Szóbeli kiváltását lehetővé tevő opcionális tétel}
Ez a feladat maximálisan 5 pontot ér. Ha ebből legalább 3 pont megvan, és az összpontszám eléri a 20, illetve 24 pontot, akkor 4-es, illetve 5-ös érdemjegyet ajánlunk.
\begin{enumerate}

\item Mik azon $x$ egészek, melyekre $x\equiv 5\pmod{6}$ és $x\equiv 7 \pmod{10}$?
\item Mik azon $x$ egészek, melyekre $x\equiv 5\pmod{6}$ és $x\equiv 8 \pmod{10}$?
\item Milyen $e, f$ egészekre oldható meg az $x\equiv e\pmod{6}$ és $x\equiv f \pmod{10}$ kongruenciarendszer? 
\item Ha az előző pontban van megoldás, akkor a megoldások halmaza egy maradékosztályt alkot-e valamilyen modulus szerint?
\item A fenti pontok mintájára fogalmazza meg a kínai maradéktétel általánosított változatát két modulus esetére olyankor, amikor a modulusok nem feltétlenül relatív prímek.

\end{enumerate}


\end{document}

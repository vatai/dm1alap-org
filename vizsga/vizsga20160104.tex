\documentclass[11pt,a4paper]{article}

\usepackage[magyar]{babel}
\usepackage[utf8]{inputenc}
\usepackage{t1enc}

\usepackage{amsmath}
\usepackage{amssymb}


\textwidth = 180mm
\hoffset = -30mm
\textheight = 265mm
\voffset = -30mm



\begin{document}

\thispagestyle{empty}

\begin{center}
\begin{large}
\noindent \textbf{Diszkrét Matematika 1.} Írásbeli vizsga, 2016. január 4. (90 perc)
\end{large}
\end{center}

{\noindent NÉV: \\ NEPTUN kód:\\ (Leendő) szakirány:\\}
\section{Alapvető fontosságú fogalmak}
A következő hat kérdésre 1-1 pont kapható. Ebből legalább 4 pontot kell szerezni.
\begin{enumerate}\setlength{\itemsep}{3cm}

\item Írja fel a szorzásra vonatkozó Moivre-azonosságot. \vspace{-.5cm}
\item Írja fel az implikáció igazságtáblázatát.
\item Szimmetrikus-e az üres halmaz mint reláció az egész számok halmazán?
\item Hányféle dobogó lehet egy 10 versenyzős futószámban (holtverseny nincs)?
\item Húzza alá a megoldható kongruenciákat: $3x\equiv 4\pmod 5$;  $30x\equiv 4\pmod 5$;  $14x\equiv 4\pmod {10}$;  $10x\equiv 1\pmod {129}$\vspace{-1.5cm}
\item Mondja ki a számelmélet alaptételét.
\end{enumerate}

\newpage
\section{Definíciók, tételkimondások}
A következő nyolc kérdésre 1-1 pont kapható. 
\begin{enumerate}\setlength{\itemsep}{4.2cm}

\item Írja fel a negyedik egységgyököket \emph{trigonometrikus} alakban.
\item Definiálja a részbenrendezést.
\item Igaz-e, hogy ha $f$ injektív és $g$ szürjektív, akkor $f\circ g$ bijektív? Indokoljon.
\item Definiálja a disztributivitást, és mutasson példát.

\newpage
\item Bontsa fel a zárójelet a binomiális tétel segítségével: $(x-y)^5$.
\item Hány olyan 7 hosszú sorozat képezhető a latin ábécé 26 eleméből, melyben csupa különböző betű szerepel (nem kell kiszámolni a konkrét számot, elég csak a képlet)?
\item Hogyan definiáljuk az Euler-féle $\phi$ függvényt?
\item Hogy szól az Euler--Fermat-tétel?


\end{enumerate}

\newpage
\section{Bizonyítások}
A következő három bizonyításra 3-3 pont kapható. Ebből legalább 3 pontot el kell érni (tételkimondásért nem jár pont).
Az összpontszám alapján a ponthatárok: 10-től 2-es, 14-től 3-as, 18-tól szóbelizhet a 4-es, illetve 5-ös osztályzatért.
\begin{enumerate}

\item Mondjon ki és bizonyítson be a halmazok komplementerének tulajdonságai közül 5-öt.
\item Mondja ki és igazolja az ismétlés nélküli kombinációk számáról szóló állítást.
\item Mondja ki és igazolja a számelmélet alaptételét (prímfelbontás létezése és egyértelműsége).

\end{enumerate}


\section{Szóbeli kiváltását lehetővé tevő opcionális tétel}
Ez a feladat maximálisan 5 pontot ér. Ha ebből legalább 3 pont megvan, és az összpontszám eléri a 20, illetve 24 pontot, akkor 4-es, illetve 5-ös érdemjegyet ajánlunk.

\begin{enumerate}

\item Melyek azok 0 és 10 közti egészek melyek felírhatók két (nem feltétlenül különböző) négyzetszám összegeként, vagyis $x^2+y^2$ alakban, ahol $x,y\in\mathbb{Z}$?
\item Igazoljuk, hogy egy $4k+3$ alakú egész szám sosem írható fel $x^2+y^2$ alakban, ahol $x, y\in\mathbb{Z}$.
\item Az alábbi három pontban azt fogjuk belátni, hogy a két négyzetszám összegeként felírható egészek halmaza zárt a szorzásra: pl.\ $5=1^2+2^2$, $13 = 2^2+3^2$, és $65 = 5\cdot 13 = 1^2+8^2$. Gauss-egésznek nevezzük azon komplex számokat, melyeknek valós és képzetes része is egész. Igazoljuk, hogy egy Gauss-egésznek a konjugáltjával vett szorzata (vagyis az abszolút értékének a négyzete) egy olyan (,,sima'' valós) egész szám, mely előáll két négyzetszám összegeként.
\item Legyen $z=a+bi$, $w=c+di$, ahol $a,b,c,d\in\mathbb{Z}$. Írjuk fel Gauss-egészként a $zw$ szorzatot, majd az előző pont alapján írjuk fel két négyzetszám összegeként ennek abszolút értékének négyzetét, vagyis az $(a^2+b^2)(c^2+d^2)$ egészet.
\item Írjuk fel két négyzetszám összegeként $41$-et, $53$-at, majd a szorzatukat, $2173$-at is.


\end{enumerate}


\end{document}

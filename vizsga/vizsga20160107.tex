\documentclass[11pt,a4paper]{article}

\usepackage[magyar]{babel}
\usepackage[utf8]{inputenc}
\usepackage{t1enc}

\usepackage{amsmath}
\usepackage{amssymb}


\textwidth = 180mm
\hoffset = -30mm
\textheight = 265mm
\voffset = -30mm



\begin{document}

\thispagestyle{empty}

\begin{center}
\begin{large}
\noindent \textbf{Diszkrét Matematika 1.} Írásbeli vizsga, 2016. január 7. (90 perc)
\end{large}
\end{center}

{\noindent NÉV: \\ NEPTUN kód:\\ (Leendő) szakirány:\\}
\section{Alapvető fontosságú fogalmak}
A következő hat kérdésre 1-1 pont kapható. Ebből legalább 4 pontot kell szerezni.
\begin{enumerate}\setlength{\itemsep}{3cm}

\item Számolja ki a következő két komplex szám reciprokát, és adja meg algebrai alakban ($a+bi$-ként): $7i$, $12-5i$.
\item Melyek injektívek az alábbi függvények közül? Húzza alá őket: négyzetre emelés a valós számok halmazán; négyzetre emelés a pozitív egészek halmazán; abszolút érték a valósok halmazán; $x\mapsto -x$ a valósakon.\vspace{-1.5cm}
\item Mikor nevezünk részbenrendezésnek egy binér relációt?
\item Hányféleképp tudunk 4 különböző csokit odaajándékozni egy 30 fős osztályból négy gyereknek (egy gyerek csak egyet kaphat)?\vspace{-1cm}
\item Bontsa fel a zárójelet: $(2x+2k)^3$. 
\item Definiálja a kitüntetett (más néven legnagyobb) közös osztó fogalmát az egészek körében.
\end{enumerate}

\newpage
\section{Definíciók, tételkimondások}
A következő nyolc kérdésre 1-1 pont kapható. 
\begin{enumerate}\setlength{\itemsep}{4.2cm}

\item Írja fel a hatványozás Moivre-féle képletét.
\item Definiálja a dichotómiát.
\item Mikor mondjuk, hogy egy $f\colon A\to B$ függvény szürjektív?
\item Definiálja a művelettartó függvény fogalmát binér műveletekre.

\newpage
\item Adjon példát asszociatív, de nem kommutatív műveletre. Adja meg az alaphalmazt is.
\item Hogy szól a polinomiális tétel?
\item Mikor mondjuk, hogy két egész relatív prím?
\item Melyek oldhatók meg az egészek körében? Húzza alá: $5x + 2y = 8$; $10x + 7y = 3z$; $57x - 12y = 14$; $10x + 35y =  100$.



\end{enumerate}

\newpage
\section{Bizonyítások}
A következő három bizonyításra 3-3 pont kapható. Ebből legalább 3 pontot el kell érni (tételkimondásért nem jár pont).
Az összpontszám alapján a ponthatárok: 10-től 2-es, 14-től 3-as, 18-tól szóbelizhet a 4-es, illetve 5-ös osztályzatért.
\begin{enumerate}

\item Mondja ki és bizonyítsa az ekvivalenciarelációk és az osztályozások kapcsolatáról szóló tételt.
\item Mondja ki és igazolja az ismétléses variációk számára vonatkozó állítást.
\item Mondja ki és igazolja az Euler--Fermat-tételt.

\end{enumerate}

\section{Szóbeli kiváltását lehetővé tevő opcionális tétel}
Ez a feladat maximálisan 5 pontot ér. Ha ebből legalább 3 pont megvan, és az összpontszám eléri a 20, illetve 24 pontot, akkor 4-es, illetve 5-ös érdemjegyet ajánlunk. Az alábbi kérdéseknél indoklást is várunk.
\begin{enumerate}

\item Nevezzünk \emph{trinzitív}nek egy $R$ binér relációt, ha minden $a, b, c, d$ esetén $(a,b), (b,c), (c,d) \in R$-ből következik $(a,d)\in R$. Melyek trinzitívek a következők közül: ,,osztója'' az egészek halmazán; ,,egyenlő'' az egészek halmazán; $\{(x,y)\mid x, y\in\mathbb{Z}, |x-y| \textrm{ páratlan}\}$ [3 pont]?
\item Igaz-e, hogy egy reláció pontosan akkor trinzitív, ha az inverze is az?
\item Igaz-e, hogy minden tranzitív reláció trinzitív? És fordítva?


\end{enumerate}


\end{document}

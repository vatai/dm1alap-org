\documentclass[11pt,a4paper]{article}

\usepackage[magyar]{babel}
\usepackage[utf8]{inputenc}
\usepackage{t1enc}

\usepackage{amsmath}
\usepackage{amssymb}


\textwidth = 180mm
\hoffset = -30mm
\textheight = 265mm
\voffset = -30mm



\begin{document}

\thispagestyle{empty}

\begin{center}
\begin{large}
\noindent \textbf{Diszkrét Matematika 1.} Írásbeli vizsga, 2016. január 21. (90 perc)
\end{large}
\end{center}

{\noindent NÉV: \\ NEPTUN kód:\\ (Leendő) szakirány:\\}
\section{Alapvető fontosságú fogalmak}
A következő hat kérdésre 1-1 pont kapható. Ebből legalább 4 pontot kell szerezni.
\begin{enumerate}\setlength{\itemsep}{3cm}

\item Mennyi $i$ abszolút értéke és argumentuma? Írja fel az $i$ szám trigonometrikus alakját.
\item Adja meg az ,,és'', a ,,vagy'' és a ,,kizáró vagy'' igazságtáblázatát.\vspace{1.5cm}
\item Mikor nevezünk ekvivalenciarelációnak egy binér relációt?\vspace{-1.5cm}
\item Hányféleképpen lehet a latin ábécé 26 betűjéből 4 hosszú sorozatokat képezni?\vspace{-1.5cm}
\item Bontsa fel a zárójelet: $(x+3y)^4$. 
\item Hány pozitív osztója van a $2^5$, illetve a $2^{100}$ számoknak?
\end{enumerate}

\newpage
\section{Definíciók, tételkimondások}
A következő nyolc kérdésre 1-1 pont kapható. 
\begin{enumerate}\setlength{\itemsep}{2.2cm}

\item Mik az $i$ harmadik gyökei? Adja meg őket trigonometrikus alakban.
\item Lehet-e egy reláció az egészek halmazán egyszerre szimmetrikus és antiszimmetrikus?
\item Mikor nevezünk egy részbenrendezést (teljes) rendezésnek?
\item Definiálja az asszociativitást.
\item Hány $k$-adosztályú ismétléses kombinációja van egy $n$ elemű halmaznak?
\item Hogy szól a szita formula 3 halmazra?
\item Definiálja az asszociáltság fogalmát.
\item Definiálja a redukált maradékrendszer fogalmát.



\end{enumerate}

\newpage
\section{Bizonyítások}
A következő három bizonyításra 3-3 pont kapható. Ebből legalább 3 pontot el kell érni (tételkimondásért nem jár pont).
Az összpontszám alapján a ponthatárok: 10-től 2-es, 14-től 3-as, 18-tól szóbelizhet a 4-es, illetve 5-ös osztályzatért.
\begin{enumerate}

\item Mondja ki és bizonyítsa a relációk inverzére vonatkozó állítást.
\item Mondja ki és igazolja a binomiális tételt.
\item Mondja ki és igazolja a logikai szitát.

\end{enumerate}

\section{Szóbeli kiváltását lehetővé tevő opcionális tétel}
Ez a feladat maximálisan 5 pontot ér. Ha ebből legalább 3 pont megvan, és az összpontszám eléri a 20, 
illetve 24 pontot, akkor 4-es, illetve 5-ös érdemjegyet ajánlunk.

Legyen $p$ egy 2-nél nagyobb prímszám. Egy $a$ egész számot kvadratikus
maradéknak neveznek modulo $p$, ha létezik olyan $x$ egész, melyre
$x^2 \equiv a \pmod p$. Ellenkező esetben kvadratikus nemmaradéknak hívjuk.

\begin{enumerate}

\item Soroljuk fel a kvadratikus maradékokat $0$ és $p$ között (a
határokat is beleértve) a $p=3$, $p=5$ és $p=7$ esetben.
\item Igazoljuk, hogy ha $a$ kvadratikus maradék modulo $p$ és
$a\equiv a' \pmod p$, akkor $a'$ is kvadratikus maradék. (Vagyis a
tulajdonság csak a maradékosztályon múlik.)
\item Igazoljuk, hogy kvadratikus maradékok szorzata, illetve nemnulla
kvadratikus maradékok reciproka is kvadratikus maradék.
\item Mit kapunk, ha egy kvadratikus nemmaradékot és egy kvadratikus
maradékot összeszorzunk?
\item Igazoljuk, hogy a nemnulla maradékosztályoknak pont a fele lesz
kvadratikus maradék.


\end{enumerate}


\end{document}

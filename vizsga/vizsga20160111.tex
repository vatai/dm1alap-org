\documentclass[11pt,a4paper]{article}

\usepackage[magyar]{babel}
\usepackage[utf8]{inputenc}
\usepackage{t1enc}

\usepackage{amsmath}
\usepackage{amssymb}


\textwidth = 180mm
\hoffset = -30mm
\textheight = 265mm
\voffset = -30mm



\begin{document}

\thispagestyle{empty}

\begin{center}
\begin{large}
\noindent \textbf{Diszkrét Matematika 1.} Írásbeli vizsga, 2016. január 11. (90 perc)
\end{large}
\end{center}

{\noindent NÉV: \\ NEPTUN kód:\\ (Leendő) szakirány:\\}
\section{Alapvető fontosságú fogalmak}
A következő hat kérdésre 1-1 pont kapható. Ebből legalább 4 pontot kell szerezni.
\begin{enumerate}\setlength{\itemsep}{3cm}

\item Írjuk fel a negyedik egységgyököket trigonometrikus és algebrai alakban.

\item Melyek antiszimmetrikusak a következő relációk közül? Jelülje aláhúzással: ,,osztója'' a természetes számok halmazán; ,,osztója'' a 10-nél kisebb természetes számok halmazán; ,,osztója'' a negatív egészek halmazán; ,,osztója'' az egészek halmazán.
\vspace{-2cm}

\item Írja fel kvantorokkal, eleme jellel és az ,,osztója'' jel használatával: minden természetes számnak van osztója.

\item Definiálja a kommutativitást (,,Egy \ldots kommutatív, ha minden \ldots'').

\item Hány olyan sorozat van, melynek hossza 5, és minden eleme A, B vagy C?

\item Definiálja a felbonhatatlanság fogalmát az \emph{egészek} körében. Felbonthatatlan-e $-5$?
\end{enumerate}

\newpage
\section{Definíciók, tételkimondások}
A következő nyolc kérdésre 1-1 pont kapható. 
\begin{enumerate}\setlength{\itemsep}{2.2cm}

\item Mit nevezünk primitív $n$-edik egységgyöknek?

\item Mennyi lesz egy primitív századik egységgyök ötvenedik hatványa?

\item Soroljon fel 4-et az unió tulajdonságai közül.

\item Definiálja binér reláció tranzitivitását.

\item Hányféleképpen  lehet 20 darab százforintost szétosztani 40 ember között (egy ember akár többet is kaphat, és mindegy, ki melyik százast kapja).

\item Ismertesse Eratoszthenész szitáját.

\item Definiálja a prím tulajdonságot.

\item Adja meg azon 1000-nél kisebb természetes számokat, melyeknek pontosan 5 pozitív egész osztójuk van.

\end{enumerate}

\newpage
\section{Bizonyítások}
A következő három bizonyításra 3-3 pont kapható. Ebből legalább 3 pontot el kell érni (tételkimondásért nem jár pont).
Az összpontszám alapján a ponthatárok: 10-től 2-es, 14-től 3-as, 18-tól szóbelizhet a 4-es, illetve 5-ös osztályzatért.
\begin{enumerate}

\item Ismertesse és igazolja a szorzásra vonatkozó Moivre-azonosságot.
\item Mondjon ki a halmazunió tulajdonságai közül ötöt, és igazolja őket.
\item Mondja ki és igazolja az ismétléses kombinációk számáról szóló állítást.

\end{enumerate}


\section{Szóbeli kiváltását lehetővé tevő opcionális tétel}
Ez a feladat maximálisan 5 pontot ér. Ha ebből legalább 3 pont megvan, és az összpontszám eléri a 20, illetve 24 pontot, akkor 4-es, illetve 5-ös érdemjegyet ajánlunk.

Az alábbi feladatok helyes megoldása által bebizonyíthatjuk, hogy végtelen sok $4k+3$, illetve $4k+1$ alakú prímszám van.

\begin{enumerate}

\item Milyen maradékot adhat egy prímszám 4-gyel osztva? Milyen maradékot ad 4-gyel osztva egy olyan szorzat, melynek tényezői páratlan prímszámok négyzetei?
\item Igazolja, hogy egy $4k+3$ alakú számnak ($k\in\mathbb{N}$) mindig van $4k+3$ alakú prímszám osztója. 
\item Legyenek $p_1, p_2, \ldots, p_n$ különböző $4k+3$ alakú prímek. Írjunk fel egy olyan $4k+3$ alakú számot, mely $p_1, p_2,\ldots, p_n$ közül egyikkel sem osztható. (Az előző pont szerint ennek van egy új $4k+3$ alakú prímosztója, így végtelen sok $4k+3$ alakú számnak kell léteznie.)
\item Bizonyítás nélkül fogadjuk el a tételt, hogy egy $n^2+1$ alakú számnak minden páratlan prímosztója $4k+1$ alakú, és ezt felhasználva, a fentiek mintájára igazoljuk végtelen sok $4k+1$ alakú prímszám létezését.



\end{enumerate}


\end{document}
